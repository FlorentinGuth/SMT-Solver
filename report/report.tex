\documentclass[a4paper, 10pt, french]{article}

\usepackage[utf8]{inputenc}
\usepackage[T1]{fontenc}
\usepackage[frenchb]{babel}
\usepackage{lmodern}
\usepackage[autolanguage]{numprint}
\usepackage{enumitem}
\usepackage{array}
\usepackage{tabularx} \newcolumntype{C}{>{\centering}X}
\usepackage{multirow}
\usepackage{hhline}
\usepackage{collcell}
\usepackage{subcaption}
\usepackage[stable]{footmisc}
\usepackage[nottoc, notlof, notlot]{tocbibind}

\usepackage[margin=2.5cm]{geometry}
\usepackage{multicol}
\usepackage[10pt]{moresize}
\usepackage{pdflscape}


\usepackage{amsthm}
\usepackage{amsmath}
\usepackage{amssymb}
\usepackage{mathrsfs}
\usepackage{amsopn}
\usepackage{stmaryrd}

\DeclareCaptionLabelFormat{listing}{Listing #2}
\DeclareCaptionSubType*[arabic]{table}
\captionsetup[subtable]{labelformat=simple}

\usepackage[langlinenos=true,newfloat=true]{minted}
\newcommand{\source}[5]{
  \begin{table}[H]
    \centering
    \small
\inputminted[frame=lines,linenos,style=colorful,fontfamily=tt,breaklines,
autogobble,firstline=#3,firstnumber=#3,lastline=#4,label={#2[#3--#4]}]{#1
}{../#2}
    
\captionsetup{name=Listing,labelformat=listing,labelsep=endash,labelfont={sc}}
    \caption{#5}
  \end{table}
  }

\newcommand{\codeOCaml}[1]{\mintinline[style=colorful,fontfamily=tt]{OCaml}{#1}}
\newcommand{\code}[1]{\texttt{#1}}
\newcommand{\foreign}[1]{\emph{#1}}



\title{Projet de Sémantique et Vérification : Solveur \foreign{SMT}}
\author{Florentin \bsc{Guth} \and Lionel \bsc{Zoubritzky}}

\begin{document}

\maketitle


\tableofcontents


\section{Liste des fichiers}

\subsection{Structures de données}

\begin{itemize}
 \item \code{union\string_find.ml} : sructure d'\foreign{union-find} classique 
(mutable),
 \item \code{persistent\string_array.ml} : implémente les tableaux non mutables 
de \cite{PersistentUF},
 \item \code{persistent\string_union\string_find.ml} : implémente la structure 
d'\foreign{union-find} non mutable de \cite{PersistentUF},
\item \code{explain\string_union\string_find.ml} : implémente un 
\foreign{union-find} permettant d'expliquer pourquoi deux éléments donnés sont 
équivalents (voir \cite{ExplainUF}),
\item \code{proof\string_union\string_find.ml} : \foreign{union-find} 
auxiliaire utilisé par l'algorithme d'explication (voir \cite{ExplainUF}).
\end{itemize}

\subsection{Fichiers auxiliaires}

\begin{itemize}
 \item \code{lexer.mll} : lexeur pour les fichiers \code{.cnfuf} (améliorés 
avec la prise en charge des fonctions),
 \item \code{parser.mly} : parseur pour les fichiers \code{.cnfuf}, réalisant 
une curryfication et un \og désenbriquement \fg\ des termes,
 \item \code{printer.ml} : fonctions d'affichages générales,
 \item \code{main.ml} : point d'entrée, lit la ligne de commande et effectue la 
résolution des fichiers donnés,
 \item \code{test\string_generator.ml} : générateur de tests aléatoires, ne 
gère ni les fonctions ni les formules non satisfiables,
 \item \code{Makefile} : \code{make all} pour compiler le solveur 
(\code{solve}) et le générateur de tests (\code{generate}), et \code{make 
clean} pour nettoyer les fichiers créés.
\end{itemize}

\subsection{Fichiers principaux}

\begin{itemize}
 \item \code{sat.ml} : solveur \foreign{SAT} prenant en entrée une formule 
booléenne, et renvoyant un assignement le cas échéant,
 \item \code{mc.ml} : ancien \foreign{Model Checker}, non incrémental,
 \item \code{incremental\string_mc.ml} : \foreign{Model Checker} incrémental, 
mais non \foreign{backtrackable},
 \item \code{smt.ml} : fichier principal réalisant l'interfaçage entre le 
solveur \foreign{SAT} et le \foreign{Model Checker}.
\end{itemize}


\section{Fonctionnement général}

\subsection{\foreign{Model Checker}}

Nous avons choisi la théorie de l'égalité, augmentée des symboles de fonction 
non-interprétés (d'arité quelconque).

Pour cela, on procède à une transformation sur la formule d'entrée afin de 
simplifier la forme des égalités :
\source{OCaml}{incremental\string_mc.mli}{7}{13}{Transformation}

On fournit une interface incrémentale, qui permet de plus de produire des 
explications pour les égalités inférées, sous forme de (petite) liste de 
fusions effectuées préalablement.
\source{OCaml}{incremental\string_mc.mli}{21}{34}{Interface du \foreign{Model 
Checker}}

Cependant, cette interface incrémentale n'est de peu d'utilité, puisqu'elle 
n'est pas persistante. Ainsi, le \foreign{backtracking} est impossible, et il 
faut tout recommencer de zéro à chaque modèle trouvé par le solveur 
\foreign{SAT}.

Les explications permettent en revanche d'ajouter des petites clauses à la 
formule à chaque appel du \foreign{Model Checker}, ce qui a pour conséquence, 
en plus d'augmenter la vitesse du solveur \foreign{SAT}, de produire des 
contre-exemples plus généraux et réduit grandement le nombre d'itérations entre 
les deux outils. \\

Concrètement, les explications sont générées en stockant, en parallèle de la 
structure d'\foreign{union-find} classique, une structure appelée \og forêt de 
preuve \fg. Celle-ci correspond à l'arbre \foreign{union-find}, mais sans 
compression de chemin, et en réalisant des fusions qui correspondent réellement 
à des égalités.


\subsection {Solveur \foreign{SAT}}

Le solveur \foreign{SAT} fonctionne en suivant les règles \foreign{DPLL(T)} avec quelques améliorations.

À chaque itération de la boucle principale du solveur, la règle \foreign{Unit} est appliquée autant que possible, puis le solveur vérifie si le modèle ainsi obtenu valide ou invalide la formule. Si un des deux cas est avéré, alors il s'arrête. Sinon, il choisit un littéral, selon l'heuristique VSIDS propoée par \cite{Chaff}, dont il devine la valeur, et il s'appelle récursivement.

L'interface proposée pour la formule et son modèle est persistante, ce qui permet de backtracker facilement en cas d'échec. Cela permet aussi d'effacer de la formule les clauses qui sont validées au fur et à mesure par le modèle. Ainsi, une clause déjà validée n'est pas explorée plusieurs fois ; elle réapparaît cependant par persistance après backtracking.

Le solveur \foreign{SAT} est entièrement incrémental : dès qu'il trouve un modèle valide, il demande au \foreign{Model Checker} une validation. Si celui-ci la lui donne, alors le problème est terminé ; sinon, il backtrack en ajoutant à sa formule l'indication du \foreign{Model Checker}, et en mettant à jour le compteur des littéraux de l'heuristique \foreign{VSIDS}.\\

Deux autres heuristiques ont été programmées (mais non raccordées au programme) : \begin{itemize}
	\item Celle notée \foreign{RAND} dans \cite{Chaff}, consistant à prendre une littéral au hasard
	\item Une autre consistant à prendre le premier littéral non assigné de la clause suivante (avec sa valeur de vérité de façon à valider cette clause)
\end{itemize}

Par ailleurs, nous avons tenté de supprimer au fur et à mesure les littéraux non validés dans les clauses, de façon à ne se préoccuper que de ceux non assignés et donc de réduire le temps d'analyse de chaque clause. Cependant, le surcoût nécessaire à la réécriture de chaque clause dépassait systématiquement le gain en temps d'analyse. Il aurait pu être plus efficace d'effectuer cette opération de nettoyage des clauses une fois de temps en temps, comme une heuristique supplémentaire, mais cela ne s'est pas vérifié dans nos tests.

\section{Améliorations}

Le caractère extrêmement brouillon de \cite{ExplainUF} a conduit à une 
multiplication de structures de données tout en ayant un code peu modulaire, et 
dont l'efficacité et la simplicité pourraient être grandement améliorées. Les 
améliorations proposées dans l'article sur la suppression de la redondance dans 
les preuves fournies n'ont pas été implémentées.

Une grande amélioration serait d'utiliser des idées similaires à celle de 
\cite{PersistentUF} afin d'obtenir un \foreign{Model Checker} totalement 
immutable, ce qui permettrait le \foreign{backtracking} et accélererait sans 
doute grandement la vitesse du solveur.

Il reste également à compléter le générateur de tests afin qu'il gère toutes 
les configurations et produise des tests plus difficiles (i.e. des formules 
vraies pour peu d'assignements).

Nous n'avons pas essayé d'implémenter un solveur \foreign{SAT} \foreign{CDCL} 
afin de comparer son efficacité au solveur \foreign{DPLL}. La méthode du \foreign{two-watched literal} n'a pas non plus été implémentée, car elle se prête assez mal à la structure de liste chaînée utilisée pour représenter les clauses.

Enfin, on pourrait imaginer que dans le cas ou la formule n'est pas satisfiable, le solveur produise une preuve de ceci, vérifiable par un programme externe comme \foreign{Coq}.


\bibliographystyle{plain}
\bibliography{report}

\end{document}
